% Chapter 6

\chapter{Analysis of Biodiesel} % Write in your own chapter title
\label{Chapter6}
\lhead{Chapter 6. \emph{Biodiesel analysis}} % Write in your own chapter title to set the page header


\section{Introduction to Biodiesel}

\section{Fatty acid profiles.}

The driving force of comprehensive 2D chromatography is orthogonality, the principle that two different separation mechanisms used successively will give greater separation power. The more different the separation mechanisms are, the higher the orthogonality. 

In \GCxGC there is always a tendency for low orthognality. This is because the basic separation mechanism of capillary chromatography is partion chromatography, and compounds with higher molecular weight tend to have longer retention times.

When using adsorption chromatography, however, the intermolecular forces are much higher, and molecular weight plays a lesser role in retention time.

A good example of this is Roger Smith's work that has proven that FAMEs separated by SFC, using pure carbion dioxide as a mobile phase and silica as a stationary phase, elute in order of number of double bonds, independent of the chain length. 